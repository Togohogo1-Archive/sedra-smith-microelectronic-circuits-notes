\documentclass{report}

\input{preamble}
\input{macros}
\input{letterfonts}

\title{\Huge{SEDRA/SMITH}\\\huge{Microelectronic Circuits}\\\normalsize{SEVENTH EDITION}}
\author{\huge{Notes by Kevin Wang}}
\date{\today}

\begin{document}

% Seems like the next time \chapter{} gets called, it updates
\setcounter{chapter}{2}

\maketitle
\newpage% or \cleardoublepage
% \pdfbookmark[<level>]{<title>}{<dest>}
\pdfbookmark[section]{\contentsname}{toc}
\tableofcontents
\pagebreak

\chapter{Semiconductors}

\section*{Introduction}
\section{Intrinsic Semiconductors}
\section{Doped Semiconductors}
\section{Current Flow in Semiconductors}
\section{The \textit{pn} Junction with an Applied Voltage}

\subsection{Qualitative Description of Junction Operation}

\subsection{The Current-Voltage Relationship of the Junction}

\subsection{Reverse Breakdown}

\section{Capacitive Effects in the \textit{pn} Junction}

2 ways charge can be stored in \textit{pn} junction.
\begin{enumerate}
	\item charge in depletion region (more visible when reverse bias)
	\item minority charge in \textit{n} and \textit{p} material (more visible when forward bias)
	\begin{itemize}
		\item concentration profile by injecting to n-dope
		\item \makebox[0pt][l]{\qquad  \qquad " \qquad  \qquad " \qquad}\phantom{concentration profile by injecting} to p-dope
	\end{itemize}
\end{enumerate}

\subsection{Depletion or Junction Capacitance}\label{sec:3.6.1-depletion-or...}

\textbf{Assumption:} \textit{pn} junction reversed bias with $V_R$, charge on either side of junction:
\begin{equation}
	Q_J = A \sqrt{2\epsilon_s q \frac{N_A N_D}{N_A + N_D} \left(V_0 + V_R\right)} = \alpha \sqrt{\left(V_0 + V_R\right)}
	\label{eq:3.6-alpha-v0-vr}
\end{equation}
We denote $\alpha$ as $A \sqrt{2\epsilon_s q \frac{N_A N_D}{N_A + N_D}}$ and observe that $Q_J \not\propto V_R$ (also not linearly related)
\begin{itemize}
	\item Hard to define capacitance that accounts for changing $Q_J$ when $V_R$ changes
\end{itemize}

\begin{figure}[!hbpt]
	\centering
	\includegraphics{path172092.png}
	\caption{The charge stored on either side of the depletion layer as a function of the reverse voltage $V_R$}
	\label{fig:charge-volt-cap-graph}
\end{figure}

\textbf{Assumption:} junction operates as a point $Q$ and define

\begin{equation}
	C_j = \frac{dQ_J}{dV_r} \bigg| _{V_R=V_Q}
	\label{eq:3.6-cap-relating-charge-to-volt}
\end{equation}
\begin{note}
	The definition of capacitance
	\begin{equation*}
		q = CV
		\implies C = q/V = \frac{\Delta q}{\Delta V}
	\end{equation*}
	\begin{itemize}
		\item Equation \ref{eq:3.6-cap-relating-charge-to-volt} useful in electronic cct design
		\item This equation used in this book frequently
		\item Called the \textbf{``incremental-capacitance approach''}
	\end{itemize}
\end{note}

Combining equations \ref{eq:3.6-cap-relating-charge-to-volt} with \ref{eq:3.6-alpha-v0-vr} we obtain:
\begin{equation}
	C_j = \frac{\alpha}{2\sqrt{V_0 + V_R}}
	\label{eq:3.6-derivative-Cj}
\end{equation}
We observe that $C_j$ at reverse bias ($V_R = 0$) is $C_{j0} = \frac{\alpha}{2\sqrt{V_0}}$, so we can write $C_j$ as
\begin{equation}
	C_j = \frac{C_{j0}}{\sqrt{1+\frac{V_R}{V_0}}}
	\label{eq:3.6-cj-wrt-cj0}
\end{equation}
Substituting for $\alpha$, we obtain:
\begin{equation}
	C_j = A \sqrt{\left(
		\frac{\epsilon_s q}{2}
	\right)\left(
		\frac{N_A N_D}{N_A + N_D}
	\right)\left(
		\frac{1}{V_0}
	\right)}
	\label{eq:3.6-cj-wrt-cj0-alpha-sub}
\end{equation}

Before leaving concept of junction capacitance, we introduce
\dfn{Terms}{
	\textbf{Abrupt junction} \textit{pn} junction, doping concentration changes abruptly at junction boundary (this is deliberately done) \\

	\textbf{Graded junction:} \textit{pn} junction, carrier concentration changes gradually from one side to another. Then $C_j$ becomes:
	\begin{equation*}
		C_j = \frac{C_{j0}}{\left(1+\frac{V_R}{V_0}\right)^m}
	\end{equation*}
	where $m$ is the \textbf{Grading coefficient}

	\begin{note}
		\begin{itemize}
			\item $m$ ranges from 1/3 to 1/2
			\item $m$ depends on manner in which concentration changes from \textit{p} to \textit{n} side
		\end{itemize}
	\end{note}
}

\qs{Exercise 3.14}{

For the \textit{pn} \dots cm$^2$. \\

\sol{Solution}
}

\subsection{Diffusion Capacitance}

\textbf{Consider:} \textit{pn} junction, forward bias: \\
\textbf{Assume:} in steady state
\begin{itemize}
	\item Minority-carrier distributions in \textit{p} and \textit{n} regions as shown in \textlangle Fig. 3.12\textrangle
		\begin{itemize}
			\item Some minority charge carrier charges stored in \textit{p} and \textit{n} regions outside depletion region
		\end{itemize}
	\item Changes in terminal voltage cause charges as mentioned $\uparrow$$\uparrow$ to change before new steady state
\end{itemize}
\textit{This}, completely different charge-storage phenomenon than \ref{sec:3.6.1-depletion-or...}
\begin{itemize}
	\item Previous section was charge-storage of non-depletion region
	\item This section is charge-storage of depletion region
\end{itemize}

We calculate excess minority-carrier charge \textlangle Fig. 3.12\textrangle by taking shaded area under exponential

\textbf{Consider:} excess hole charges in \textit{n} region $Q_p$
\begin{align}
	Q_p &= Aq \times \text{shaded area under the $p_n(x)$ curve}\notag \\
		&= Aq \left[p_n(x_n) - p_{n0}\right]L_p \label{eq:3.6-area-under-curve-charge}
\end{align}
\begin{note}
	Recall area under exponential curve $Ae^{-x/B}$ is equal to AB
\end{note}

Doing some substitutions \textlangle add sections\textrangle to Eq. (\ref{eq:3.6-area-under-curve-charge}):
\begin{equation}
	Q_p = \frac{L_p^2}{D_p}I_p
\end{equation}
We note that the factor $\frac{L_p^2}{D_p}$ relates $Q_p$ to $I_p$ is very useful parameter, and has dimensions of time (s). Thus we denote:
\begin{equation}
	\tau_p = \frac{L_p^2}{D_p}
\end{equation}
So:
\begin{equation}
	Q_p = \tau_p I_p \label{eq:3.6-qptpip}
\end{equation}

\dfn{Terms}{
	\textbf{Minority-carrier (hole) lifetime}: Average time it takes for a hole injected into the n region to recombine with a majority electron, denoted $\tau_p$ \\

	This definition has the following implications:
	\begin{itemize}
		\item Entire charge, $Q_p$ disappears
		\item $Q_p$ has to be replenished every $\tau_p$ seconds
		\item The current responsible for replenishing is $I_p$
	\end{itemize}
}

Similarly for electrons charge in \textit{p} region:
\begin{equation}
	Q_n = \tau_n I_n \label{eq:3.6-qntnin}
\end{equation}
Where $\tau_n$ is electron lifespan in \textit{p} region. Thus, the total excess minority-carrier charge:
\begin{equation}
	Q = \tau_p I_p + \tau_n I_n
	\label{eq:3.6-qtpiptnin}
\end{equation}
In terms of $I = I_p + I_n$, the diode current
\begin{equation}
	Q = \tau_T I
\end{equation}

\dfn{Term}{
	\textbf{Mean transit time}: For the junction, is equal to $\tau_T$ \\

	We recognize that one side of junction more heavily doped than another. If $N_A >> N_D$:
	\begin{itemize}
		\item $I_p$~\textgreater{}\textgreater~$I_n$
		\item $I \approx I_p$
		\item $Q_p$~\textgreater{}\textgreater~$Q_n$
		\item $Q \approx Q_p$
		\item $\bm{\tau_T \approx \tau_p}$
	\end{itemize}
}

\dfn{Term}{
	\textbf{Incremental diffusion capacitance}: Defined $C_d$, for small changes around a bias point:
	\begin{equation}
		C_d = \frac{dQ}{dV} = \left(\frac{\tau_T}{V_T}\right) I
	\end{equation}
	Where $I$ is the forward-bias current \\

	Note:
	\begin{itemize}
		\item $C_d \propto I$
			\begin{itemize}
				\item Because of this, $C_d$ negligibly small when reverse bias
			\end{itemize}
		\item To keep $C_d$ small, transit time must be small
			\begin{itemize}
				\item Important requirement for \textit{pn} junction for high-speed or high-frequency
			\end{itemize}
	\end{itemize}

}

\qs{Exercise 3.15}{

Use the definition \dots

\sol{Solution}
}

\qs{Exercise 3.16}{

For the \textit{pn} \dots

\sol{Solution}
}

\end{document}
